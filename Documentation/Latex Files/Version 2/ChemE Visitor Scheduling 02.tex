\author{Garrett Dowdy}
\title{ChemE Visitor Scheduling 2}
\date{\today}

\documentclass[12pt]{article}
\usepackage[margin=0.5in]{geometry}

\usepackage{hyperref}

\usepackage{amsmath}
\usepackage{amsfonts}
\usepackage{amssymb}
\usepackage{amsthm}

\usepackage{graphicx}
\usepackage{epstopdf}
\usepackage{float}
\usepackage{accents} %required for underbar
\newcommand{\ubar}[1]{\underaccent{\bar}{#1}}

\newcommand{\R}{\mathbb R}
\newcommand{\mbf}[1]{\mathbf{#1}}
\newcommand{\trans}{^\text{T}}
\DeclareMathOperator{\Tr}{Tr}

%define Theorem environments:
\newtheorem{theorem}{Theorem} %[subsection]
\newtheorem{assumption}{Assumption} %[subsection]
\newtheorem{claim}{Claim}%[subsection]
\newtheorem{corollary}[claim]{Corollary}
\newtheorem{lemma}{Lemma} %[subsection]
\theoremstyle{definition}
\newtheorem{definition}{Definition} %[subsection]
\newtheorem{definition set}{Definition Set}%[subsection]
\newtheorem{problem statement}{Problem Statement} %[subsection]
\theoremstyle{remark}
\newtheorem{remark}{Remark}%[subsection]
\theoremstyle{remark}
\newtheorem{proposition}{Proposition}%[subsection]

%code for inserting EPS figures
%\begin{figure}[H]
%    \centering
%    \includegraphics[scale = 0.75]{H1coneCropped}
%    \caption{The cone of points satisfying $\mathbf H_1(\mathbf y) \succeq \mathbf 0$.}
%    \label{fig: H1 cone}
%\end{figure}

\usepackage[utf8]{inputenc}
\usepackage{csquotes}


\begin{document}

\maketitle

\section{Introduction}
This document describes a mathematical model for solving the ChemE Visitor Scheduling problem.

My understanding of the problem is as follows:
\begin{itemize}
\item
Each visitor submits a rank-ordered list of the faculty they\rq{}d like to meet.

\item
Each faculty member submits their availability.

\item
Visitors are assigned to meet with faculty during specific periods.

\item
We want to maximize the overall happiness of the visitors.

\item
We want to maximize the minimum level of happiness.
\end{itemize}

\section{Definitions}
\subsection{Sets}
\begin{itemize}
\item
Let $V = \{1,...,n_V\}$ denote the set of visitors.

\item
Let $F = \{1,...,n_F\}$ denote the set of faculty.

\item
Let $T = \{1,...,n_T\}$ denote the set of time periods.
\end{itemize}

\subsection{Problem Data}
\begin{itemize}
\item
Let $A \in \{0,1\}^{n_F \times n_T}$ be a matrix describing the availability of the faculty.
In particular, if faculty $f$ is available for a meeting during period $t$, then $A(f,t) = 1$.
Otherwise, $A(f,t) = 0$.

\item
Let $P \in \{0, \hdots  , 10\}^{n_V \times n_F}$ be a matrix describing the visitors\rq{} preferences regarding who they\rq{}d like to meet.
In particular, if visitor $v$ really wants to meet faculty $f$ (that is, if faculty $f$ is their first choice), then $P(v,f) = 10$.
If faculty $f$ is their second choice, $P(v,f) = 9$, and so on, down to their tenth choice, for which $P(v,f) = 1$.
For a faculty member $f$ not appearing on visitor $v$\rq{}s preference list, $P(v,f) = 0$.
\end{itemize}

\subsection{Decision Variables}
\begin{itemize}
\item
Let $x \in \{0,1\}^{n_V \times n_F \times n_T}$ be a matrix describing decisions to schedule meetings between visitors and faculty.
In particular, if visitor $v$ is assigned to meet with faculty $f$ during period $t$, then $x(v,f,t) = 1$.
Otherwise, $x(v,f,t) = 0$.

\item
Let $y \in \{0,1\}^{n_V \times n_T}$ be a matrix describing decisions to schedule the visitors for free time.
In particular, if visitor $v$ is scheduled for free time during period $t$, then $y(v,t) = 1$.
Otherwise, $y(v,t) = 0$.

\item
Let $h \in \R$ be the minimum level of happiness.
This definition will be enforced via a combination of the objective function and constraints.
\end{itemize}

\section{Constraints}
\begin{enumerate}
\item
Each visitor must be assigned to either a faculty meeting or free time for each period:
\begin{equation}
\sum_{f \in F} x(v,f,t) + y(v,t) = 1, \ \ \ \forall v \in V, \forall t \in T.
\end{equation}

\item
Each faculty can meet with at most one student during a given period:
\begin{equation}
\sum_{v \in V} x(v,f,t) \leq 1, \ \ \ \forall f \in F, \forall t \in T.
\end{equation}

\item
Visitors can only meet with faculty when the faculty are available:
\begin{equation}
\sum_{v \in V} x(v,f,t) \leq A(f,t), \ \ \ \forall f \in F, \forall t \in T.
\end{equation}

\item
Each visitor-faculty pair can only meet at most once
\begin{equation}
\sum_{t \in T} x(v,f,t) \leq 1, \ \ \ \forall v \in V, \forall f \in F.
\end{equation}

\item (Optional) Each visitor must have at least $n^L$ periods of free time:
\begin{equation}
\sum_{t \in T} y(v,t) \geq n^L, \ \ \ \forall v \in V.
\end{equation}

\item (Optional) Each visitor can have at most $n^U$ periods of free time:
\begin{equation}
\sum_{t \in T} y(v,t) \leq n^U, \ \ \ \forall v \in V.
\end{equation}

\item 
Each visitor\rq{}s happiness level must be at least $h$:
\begin{equation}
\sum_{t \in T} \sum_{f \in F} x(v,f,t) P(v,f) \geq h, \ \ \ \forall v \in V.
\end{equation}
\end{enumerate}

\section{Objective}
The objective function (to be maximized) will have three terms:
\begin{equation}
c_1 \cdot x + c_2 \cdot x + c_3 h
\end{equation}
The rationale for each of these terms is described below.
\begin{itemize}
\item
Give the visitors their preferences:
\begin{equation}
c_1(v,f,t) = P(v,f), \ \ \ \forall v \in V, \forall f \in F, \forall t \in T,
\end{equation}

\item
It\rq{}s better to have a meeting than no meeting at all:
\begin{equation}
c_2(v,f,t) = w_{\text{meet}}, \ \ \ \forall v \in V, \forall f \in F, \forall t \in T,
\end{equation}
The weighting factor $w_{\text{meet}} \geq 0$ determines how important this objective is.
I recommend that its value be kept small, for example, $w_{\text{meet}} = 1$.

\item
Maximize the minimum level of happiness:
\begin{equation}
c_3 = w_{\text{min}}.
\end{equation}
The weighting factor $w_{\text{min}} \geq 0$ determines how important this objective is.
I recommend that its value be kept small, for example, $w_{\text{min}} = 1$.
\end{itemize}


\section{The Full Problem}
The full optimization problem is:
\begin{equation}
\begin{aligned}
&\max_{x,y} && c_1 \cdot x + c_2 \cdot x + c_3 h\\
&\ \ \text{s.t} && x \in \{0,1\}^{n_V \times n_F \times n_T}, \\
& && y \in \{0,1\}^{n_V \times n_T}, \\
& && \sum_{f \in F} x(v,f,t) + y(v,t) = 1, \ \ \ \forall v \in V, \forall t \in T, \\
& && \sum_{v \in V} x(v,f,t) \leq 1, \ \ \ \forall f \in F, \forall t \in T, \\
& && \sum_{v \in V} x(v,f,t) \leq A_V(f,t), \ \ \ \forall f \in F, \forall t \in T, \\
& && \sum_{t \in T} x(v,f,t) \leq 1, \ \ \ \forall v \in V, \forall f \in F, \\
& && \sum_{t \in T} y(v,t) \geq n^L, \ \ \ \forall v \in V, \\
& && \sum_{t \in T} y(v,t) \leq n^U, \ \ \ \forall v \in V, \\
& && \sum_{t \in T} \sum_{f \in F} x(v,f,t) P(v,f) \geq h, \ \ \ \forall v \in V.
\end{aligned}
\end{equation}



\end{document}